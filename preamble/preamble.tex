% load preamble-min, and apply geometry
% ----------------------------------------------------------------------
% Check options
% ----------------------------------------------------------------------
\makeatletter%
\newif\ifprintformat
\@ifclasswith{book}{printformat}{\printformattrue}{\printformatfalse}
\newif\ifcheckreferences
\@ifclasswith{book}{checkreferences}{\checkreferencestrue}{\checkreferencesfalse}
\newif\ifgrayfigures
\@ifclasswith{book}{grayfigures}{\grayfigurestrue}{\grayfiguresfalse}
\makeatother
\ifgrayfigures
\def\figpref{gray-}
\else
\def\figpref{}
\fi
% Line spacing
% ----------------------------------------------------------------------
\usepackage{setspace}
\renewcommand{\baselinestretch}{1.0}% default line spacing, up to 1.2 looks nice
%
% Geometry def
% ----------------------------------------------------------------------
% good ref: http://practicaltypography.com
%			and: http://mirrors.ctan.org/macros/latex/contrib/geometry/geometry.pdf
%
\ifdefined\thegeometrytextwidth
\else
	\def\thegeometrytextwidth{6.0in}
\fi
\ifdefined\thegeometryinit
\else
	\def\thegeometryinit{paper=a4paper,headheight=16pt,marginparwidth=60pt,textwidth=\thegeometrytextwidth,textheight=8.5in,heightrounded,vmarginratio=8:10}
\fi
%
\ifdefined\thegeometryextra
\else
	%\ifprintformat
		%\def\thegeometryextra{hmarginratio=3:4,bindingoffset=0.5cm}% margins for printing
	%\else
		\def\thegeometryextra{hcentering,bindingoffset=0cm}% margins for electronic document
	%\fi
\fi
%
\def\thegeometry{\thegeometryinit,\thegeometryextra}
%
% BLANK PAGE
% ----------------------------------------------------------------------
\newcounter{totalblankpages}
\setcounter{totalblankpages}{0}% set it to something just in case
\newcounter{mypagecount}% create a new counter
\setcounter{mypagecount}{0}% set it to something just in case
\newenvironment{interlude}{% create a new environment for the unnumbered section(s)
\clearpage
\setcounter{mypagecount}{\value{page}}% use the new counter we created to hold the page count at the start of the unnumbered section
\thispagestyle{empty}% we want this page to be empty (adjust to use a modified page style)
\pagestyle{empty}% use the same style for subsequent pages in the unnumbered section
}{%
\clearpage
\setcounter{page}{\value{mypagecount}}% restore the incremented value to the official tally of pages so the page numbering continues correctly
}
\ifprintformat
	\newcommand\blankpage{%
	\begin{interlude}
	\null
	\stepcounter{totalblankpages}
	\end{interlude}
	}
\else
	\newcommand\blankpage{%
	%
	}
\fi

\usepackage[\thegeometry]{geometry}
\usepackage{layout}% to get and/or display page layout details
%
% Fonts
% ----------------------------------------------------------------------
\usepackage{iftex}
\ifPDFTeX
	\usepackage{cmap}% make pdf files searchable and copyable
	\usepackage[utf8]{inputenc}
	\usepackage[T1]{fontenc}
	\usepackage{lmodern}
	\usepackage{amsmath,amsfonts,amssymb}
	%
	\usepackage{isomath}% english standards
	%
	\def\myfancyfont{}
\else
	\usepackage{amsmath,amssymb}
	\usepackage[no-math]{fontspec}
	%
	\defaultfontfeatures{Ligatures=TeX}
	%
	% If latin modern fonts are not found using XeTex in linux:
	%	- [ubuntu]Make sure lmodern package is installed
	%	- [ubuntu] create a new file 09-texmffonts.conf in /etc/fonts/conf.avail/ and sym link it to ../conf.d
	%	- [ubuntu] put in that file the directories /usr/share/texmf/fonts/opentype|truetype|type1 and /usr/share/texlive/texmf-dist/fonts/opentype|truetype|type1
	%	- [arch] Install the aur package otf-latin-modern
	%	- [arch] This should be optional and the config should already be there, otherwise put directories might be instead /usr/share/texmf-dist/fonts/opentype|truetype|type1 and /usr/local/share/texmf/fonts/opentype|truetype|type1
	%	- [arch] Make sure the sym link to the tex font config exists in /etc/fonts/conf.d
	%	- If you have added a font conf then run fc-cache -f -v
	\setmainfont[
		Numbers=Proportional,
		Ligatures={TeX},
		SlantedFont={Latin Modern Roman Slanted},
		SmallCapsFont={Latin Modern Roman Caps},
		ItalicFeatures={SmallCapsFeatures={FakeSlant=0.2}}
	]{Latin Modern Roman}
	\setsansfont[
		Scale=MatchLowercase
	]{Latin Modern Sans}% Note that there is no latin modern Sans Caps font nor slanted one.
	\setmonofont[
		Scale=MatchLowercase
	]{Latin Modern Mono Light}
	%
	%\usepackage[math-style=ISO,bold-style=ISO]{unicode-math}% english standards 
	\usepackage[math-style=french]{unicode-math}% french standards 
	\setmathfont[Path=./fonts/math/]{latinmodern-math.otf}
	%
	\ifLuaTeX
	\else
	\fi
	%
	\newfontfamily\myfancyfont[
		ItalicFont=EBGaramond12-Italic.otf,
		Ligatures={Rare, Historic, TeX},
		Contextuals=Alternate,
		RawFeature={+dlig},
		ItalicFeatures={Style=Swash}
	]{EBGaramond12-Regular.otf}
\fi

% Some maths defs
% ----------------------------------------------------------------------
\newcommand*\diff{\mathop{}\!\mathrm{d}}
\newcommand*\Diff[1]{\mathop{}\!\mathrm{d^#1}}

% Miscellaneous - 1
% ----------------------------------------------------------------------
%\usepackage{flafter}% forces floats to appear only after the place where they are defined
\usepackage{verbatim}% for block comment
\usepackage{graphicx}
\graphicspath{{./figures/}}
\usepackage{booktabs,tabularx}% better column width for tabular environment
\usepackage{multirow}% multirow env for tabular
\usepackage{xcolor}
%\usepackage{tikz}% used in cover page
%\usetikzlibrary{calc}% needed for coordinate calculation in tikz
\usepackage{tcolorbox}% used in cover page
\tcbuselibrary{skins}% lib for tcolorbox
\usepackage{shadowtext}% used in cover page
%\usepackage{listings}% code source formatting
\usepackage{algorithm}% float wrapper
\usepackage{algorithmicx}% typesetting environment
\usepackage[noend]{algpseudocode}% layout for algorithmicx
\usepackage{cite}
\usepackage{IEEEtrantools}% needed to use \bstctlcite with IEEEtran
\ifcheckreferences
	\usepackage[ignoreunlbld]{refcheck}% checks for unused citations and more 
\fi
\usepackage[bottom]{footmisc}% make sure foot notes have constant distance with fancy foot when using raggedbottom 
\usepackage{fancyhdr}% fancy headers and footers
%\usepackage{shorttoc}% short table of contents
%\usepackage[tiny, md, sc]{titlesec}
\usepackage{titlesec}
\pagestyle{plain}
\usepackage{titling}% title, author, date and thanks commands
\usepackage[Conny]{fncychap}% nice chap titles: Lenny, Conny, Bjornstrup
%\usepackage[palatino]{quotchap}
\usepackage{lettrine}% lettrine command
\usepackage[titletoc]{appendix}
\usepackage[tight,nice]{units}% provides commands: \unit, \unitfrac and \nicefrac
\usepackage[titles]{tocloft}% table of content format
\setlength{\cftbeforesecskip}{5pt}
%\renewcommand{\cftchapafterpnum}{\vspace{5pt}}
\renewcommand{\cftsecleader}{\cftdotfill{\cftdotsep}}
\renewcommand{\cftchapfont}{\normalsize \scshape}
\renewcommand{\cftchappagefont}{\normalsize \scshape}
%
\begingroup\expandafter\expandafter\expandafter\endgroup
\expandafter\ifx\csname IncludeInRelease\endcsname\relax
  \usepackage{fixltx2e}% include this only for releases prior to 2015
\fi
%
\usepackage{ragged2e}
%
\parindent 15pt
%
\widowpenalty=9000
\clubpenalty=9000
%\brokenpenalty=501
\raggedbottom% make all pages the height of the text on that page. No need for extra vertical space (especially between section title and text).
%
%\renewcommand{\thefootnote}{\fnsymbol{footnote}}
%
\hyphenation{net-works}% correct bad hyphenation here
%
\setcounter{secnumdepth}{3}% 1 = sections only, 2 = sections + subsections, 3 = sections + subsections + subsubsections
\setcounter{tocdepth}{2}
%
% hyperref related packages
% ----------------------------------------------------------------------
% note: You might need to add hyperindex=false if "see" and "seealso" commands are not working in index generation.
% However you should not need this, as it should have been fixed in ./preamble/index.xdy
% See: https://en.wikibooks.org/wiki/Talk:LaTeX/Indexing#Texindy.2C_hyperref_and_textbf.2C_textit_modifiers
% and see: http://geekographie.maieul.net/190
%\usepackage[hidelinks,hyperindex=false]{hyperref}
\usepackage[hidelinks]{hyperref}
\usepackage{bookmark}
%
% Acronyms
% ----------------------------------------------------------------------
\usepackage[xindy={language=french, codepage=utf8}, style=altlist, nonumberlist]{glossaries}% glossaries
\usepackage{glossary-mcols}
\renewcommand*{\acronymtype}{acronym}
\newglossary[alg]{acronym}{acr}{acn}{\acronymname}
\makeglossaries
%
% locale related packages
% ----------------------------------------------------------------------
\usepackage[labelfont={footnotesize,up,bf,sf,singlespacing},
			textfont={footnotesize,up,md,sf,singlespacing},
			justification={justified},
			singlelinecheck=false,
			margin=0pt,
			figurewithin=chapter,
			tablewithin=chapter]{caption}% caption package which sets figure, table caption font, format, name etc.
%
\usepackage{polyglossia}
\setmainlanguage{french}
\setotherlanguage{english}
% We need arabic locale for abstract-ar, however we do not load it using polyglossia. It loads bidi package which breaks fncychap and maybe other packages.
%
% Miscellaneous - 2
% ----------------------------------------------------------------------
% 
% Definitions Commands
% ----------------------------------------------------------------------
\makeatletter
\ifdefined\theauthor
	\def\@author{\theauthor}
\else
	\def\author#1{\gdef\@author{#1}}
	\def\theauthor{\@author}
\fi
\def\titlefrench#1{\gdef\@titlefrench{#1}}
\def\thetitlefrench{\@titlefrench}
\def\titleenglish#1{\gdef\@titleenglish{#1}}
\def\thetitleenglish{\@titleenglish}
\def\keywordsfrench#1{\gdef\@keywordsfrench{#1}}
\def\thekeywordsfrench{\@keywordsfrench}
\def\keywordsenglish#1{\gdef\@keywordsenglish{#1}}
\def\thekeywordsenglish{\@keywordsenglish}
\def\advisor#1{\gdef\@advisor{#1}}
\def\theadvisor{\@advisor}
\def\degreeyear#1{\gdef\@degreeyear{#1}}
\def\thedegreeyear{\@degreeyear}
\def\degreemonth#1{\gdef\@degreemonth{#1}}
\def\thedegreemonth{\@degreemonth}
\def\degreeday#1{\gdef\@degreeday{#1}}
\def\thedegreeday{\@degreeday}
\makeatother


%
% Styling Commands
% ----------------------------------------------------------------------
%

\definecolor{chaptergrey}{rgb}{0.1000, 0.1600, 0.0060}%
\definecolor{hdrrule}{rgb}{0.4000, 0.4000, 0.4000}%

\fancypagestyle{empty}{
	\fancyhf{}
	\cfoot{}%\thepage}
	\renewcommand{\headrulewidth}{0pt}% header horizontal line
}

\fancypagestyle{emptynonum}{
	\fancyhf{}
	\cfoot{}
	\renewcommand{\headrulewidth}{0pt}% header horizontal line
}

\makeatletter
\@ifclasswith{book}{openany}{%
	\renewcommand*{\cleardoublepage}{\clearpage}%
}{%
	\renewcommand*{\cleardoublepage}{
		\clearpage
		\if@twoside
			\ifodd
				\c@page
			\else
				\hbox{}%
				\thispagestyle{empty}%
				
				\newpage%
				\if@twocolumn\hbox{}
					\newpage
				\fi
			\fi
		\fi
	}%
}
\makeatother

% page header
% ----------------------------------------------------------------------
\fancyhf{}% clear header
%\fancyhead[RO,LE]{\thepage}
%\fancyhead[LO]{\normalfont\nouppercase{\rightmark}}
%\fancyhead[RE]{\normalfont\nouppercase{\leftmark}}

\newcounter{realpagenum}
\fancyhead[RO,RE]{%
\setcounter{realpagenum}{\value{page}}%
\addtocounter{realpagenum}{\value{totalblankpages}}%
\ifodd\value{realpagenum}{\thepage}\else{\normalfont\nouppercase{\leftmark}}\fi%
}
\fancyhead[LO,LE]{%
\setcounter{realpagenum}{\value{page}}%
\addtocounter{realpagenum}{\value{totalblankpages}}%
\ifodd\value{realpagenum}{\normalfont\nouppercase{\rightmark}}\else{\thepage}\fi%
}

\pretocmd{\headrule}{\color{hdrrule}}{}{}
%\renewcommand{\headrulewidth}{0pt}% no horizontal line in headers
%\renewcommand{\footrulewidth}{0.4pt}% footer horizontal line
\pagestyle{fancy}
%
%\def\markchapter#1{\markboth{\MakeUppercase{#1}}{\MakeUppercase{#1}}}
\def\markchapter#1{\markboth{#1}{#1}}
\def\addchapter#1{\cleardoublepage\phantomsection{}\markchapter{#1}}% adds chapter but does not list it in toc
\def\addchaptertoc#1{\addchapter{#1}\addcontentsline{toc}{chapter}{#1}}% adds chapter and lists it in toc
\pretocmd{\tableofcontents}{\addchaptertoc{\contentsname}}{}{}
\newcommand{\listspart}{\addcontentsline{toc}{chapter}{Listes}}
\newif\ifListsStarted
\ListsStartedfalse
\newcommand{\listsname}{Listes}
\newcommand{\checkListsStarted}{\ifListsStarted\else\addcontentsline{toc}{chapter}{\listsname}\ListsStartedtrue\fi}
\pretocmd{\listoffigures}{\addchapter{\listfigurename}\checkListsStarted\addcontentsline{toc}{section}{\listfigurename}}{}{}
\pretocmd{\listoftables}{\addchapter{\listtablename}\checkListsStarted\addcontentsline{toc}{section}{\listtablename}}{}{}
\pretocmd{\printnomenclature}{\addchapter{\nomname}\checkListsStarted\addcontentsline{toc}{section}{\nomname}}{}{}
\newcommand\listacronymname{Liste des abréviations, sigles et acronymes}
\newcommand\listofacronyms{\addchapter{\listacronymname}\checkListsStarted\addcontentsline{toc}{section}{\listacronymname}\printglossary[type=\acronymtype,title=\listacronymname,style=mcoltree,nogroupskip]}
\pretocmd{\printindex}{\addchaptertoc{\indexname}}{}{}
\pretocmd{\bibliography}{\addchaptertoc{\bibname}}{}{}
\makeatletter
\newcounter{ChapterCounter}
\renewcommand{\theHsection}{\theChapterCounter.\the\value{section}}% used by hyperref, we modified this to make sure we provide unique string when using chapter*
\newif\ifChapterIsStarred
\renewcommand{\thesection}{\ifChapterIsStarred\else\the\value{chapter}.\fi\the\value{section}}% we modified this to hide chapter number when using chapter*
\def\startschapternotext#1{%
	\ChapterIsStarredtrue%
	\stepcounter{ChapterCounter}%
	\phantomsection\markchapter{#1}%
	\setcounter{section}{0}%
}
\pretocmd{\@schapter}{%
	\startschapternotext{#1}
}{}{}
\pretocmd{\@chapter}{%
	\ChapterIsStarredfalse%
	\stepcounter{ChapterCounter}%
}{}{}
\if@twoside
	% chapter title in even pages and section in odd pages
	\renewcommand{\chaptermark}[1]{\markboth{\chaptername\ \thechapter.\ #1}{\chaptername\ \thechapter.\ #1}}
	\renewcommand{\sectionmark}[1]{\markright{\thesection\ #1}}
\else
	% chapter title in both odd and even pages and no section
	\renewcommand{\chaptermark}[1]{\markboth{\chaptername\ \thechapter.\ #1}{\chaptername\ \thechapter.\ #1}}
	\renewcommand{\sectionmark}[1]{}
\fi
\makeatother

% front style
% ----------------------------------------------------------------------
\renewcommand{\frontmatter}{
	\pagenumbering{roman}
}

% main style
% ----------------------------------------------------------------------
\renewcommand{\mainmatter}{
	\cleardoublepage
	\setcounter{page}{1}
	\pagenumbering{arabic}
}

% back style
% ----------------------------------------------------------------------
\renewcommand{\backmatter}{
	\cleardoublepage
}

% etc
% ----------------------------------------------------------------------

\newcommand{\nullpart}{
	\bookmarksetup{startatroot}% Use bookmark package so that all following chapters appear in the root (no part)
}

\newenvironment{copyrightpage}{
	\clearpage{}% appears at the verso of the title page with year, author, the publisher, the ISSN number, etc. 
	\null\vspace*{\fill}
	\begin{center}
}{
	\end{center}
	\null
}

\newenvironment{dedications}{
	\cleardoublepage
	\null\vspace*{\stretch{1}}
	\begin{flushright}
	\large\itshape\myfancyfont
}{
	\end{flushright}
	\vspace*{\stretch{2}}\null
}

\newenvironment{abstract}{
	\par\noindent\textbf{\small\abstractname}%
	\par%
}{
}

\def\acknowledgmentsname{Remerciements}
\newenvironment{acknowledgments}{
	\chapter*{\acknowledgmentsname}
	\addcontentsline{toc}{chapter}{\acknowledgmentsname}
	\setstretch{1.2}
	\large\itshape
}{
}

\def\prefacename{Préface}
\newenvironment{preface}{
	\chapter*{\prefacename}
	\addcontentsline{toc}{chapter}{\prefacename}
}{
}

\def\listofpublicationsname{Nos publications}
\newenvironment{listofpublications}{
	\chapter*{\listofpublicationsname}
	\addcontentsline{toc}{chapter}{\listofpublicationsname}
}{
}

\newcommand{\inenglish}[1]{\textenglish{\textit{#1}}}% foreign langage for french should be in italic

\makeatletter
\def\setfigpath#1{\gdef\@figpath{#1}}
\def\figpath{\@figpath}
\makeatother

\captionsetup{%
	figurename=Figure,
	tablename=Tableau
}
%\AtBeginDocument{%
%\renewcommand\figurename{\textsc{Figure}}
%\renewcommand\tablename{\textsc{Tableau}}
%}

% Algorithms
% ----------------------------------------------------------------------
% numbering
\makeatletter 
\renewcommand\thealgorithm{\thechapter.\arabic{algorithm}} 
\@addtoreset{algorithm}{chapter} 
\makeatother

% locale
\makeatletter
\newcommand{\newalgname}[1]{%
  \renewcommand{\ALG@name}{#1}%
}
\newalgname{Algorithme}
\renewcommand{\listalgorithmname}{Liste des \ALG@name s}
\makeatother

\renewcommand{\algorithmicrequire}{\textbf{Entrée:}}
\renewcommand{\algorithmicensure}{\textbf{Sortie:}}
\renewcommand{\algorithmicend}{\textbf{fin}}
\renewcommand{\algorithmicif}{\textbf{si}}
\renewcommand{\algorithmicthen}{\textbf{alors}}
\renewcommand{\algorithmicelse}{\textbf{sinon}}
\renewcommand{\algorithmicfor}{\textbf{pour}}
\renewcommand{\algorithmicforall}{\textbf{pour chaque}}
\renewcommand{\algorithmicdo}{\textbf{faire}}
\renewcommand{\algorithmicwhile}{\textbf{tant que}}
\renewcommand{\algorithmicrepeat}{\textbf{répéter}}
\renewcommand{\algorithmicuntil}{\textbf{jusqu'à}}
\renewcommand{\algorithmicreturn}{\textbf{retourner}}
\newcommand{\algorithmicelsif}{\algorithmicelse\ \algorithmicif}
\newcommand{\algorithmicendif}{\algorithmicend\ \algorithmicif}
\newcommand{\algorithmicendfor}{\algorithmicend\ \algorithmicfor}


% other styling
% ----------------------------------------------------------------------

\newcommand{\enquote}[1]{«\,#1\,»}

\newcommand{\manualpagebreak}{\pagebreak}

% Copyright
% ----------------------------------------------------------------------
%\begin{copyrightpage}
%	\copyright{} \thedegreeyear{} \theauthor{}
%
%	TOUS DROITS RÉSERVÉS
%\end{copyrightpage}
% definitions
%
% Commands
% ----------------------------------------------------------------------
\makeatletter
\ifdefined\theauthor
	\def\@author{\theauthor}
\else
	\def\author#1{\gdef\@author{#1}}
	\def\theauthor{\@author}
\fi
\def\titlefrench#1{\gdef\@titlefrench{#1}}
\def\thetitlefrench{\@titlefrench}
\def\titleenglish#1{\gdef\@titleenglish{#1}}
\def\thetitleenglish{\@titleenglish}
\def\keywordsfrench#1{\gdef\@keywordsfrench{#1}}
\def\thekeywordsfrench{\@keywordsfrench}
\def\keywordsenglish#1{\gdef\@keywordsenglish{#1}}
\def\thekeywordsenglish{\@keywordsenglish}
\def\advisor#1{\gdef\@advisor{#1}}
\def\theadvisor{\@advisor}
\def\degreeyear#1{\gdef\@degreeyear{#1}}
\def\thedegreeyear{\@degreeyear}
\def\degreemonth#1{\gdef\@degreemonth{#1}}
\def\thedegreemonth{\@degreemonth}
\def\degreeday#1{\gdef\@degreeday{#1}}
\def\thedegreeday{\@degreeday}
\makeatother
%
% Values
% ----------------------------------------------------------------------
\date{\today}% define date as today
\titlefrench{Le titre du mémoire (modifiable dans frontmatter/definitions.tex)}
\keywordsfrench{Mot clé 1, Mot clé 2, Mot clé 3, Mot clé 4, Mot clé 5}
\titleenglish{The titre of the document in english}
\keywordsenglish{Keyword 1, Keyword 2, Keyword 3, Keyword 4, Keyword 5}
\author{Prénom NOM \textnormal{et} Prénom NOM}
\advisor{Prénom NOM}

% ... about the degree.
\degreeyear{2024}
\degreemonth{mai}
\degreeday{26}

% Acronyms
%

% Ajouter les acronymes utilisés dans le document ici
%
\newacronym{CPU}{CPU}{\inenglish{Central Processing Unit}}
\newacronym{TIC}{TIC}{Technologies de l'Information et des Communications}
\newacronym{VIM}{VIM}{\inenglish{Virtual Infrastructure Manager}}
\newacronym{VM}{VM}{Machine Virtuelle}

\glsaddall% add all glossary/acronym entries. Instead we could call the \gls{} command where appropriate.

% Set pdf meta data
\hypersetup{
	pdfencoding=auto,
    pdfauthor={\theauthor},
    pdftitle={\thetitlefrench},
    pdfkeywords={\thekeywordsfrench}
}



