% 
% Definitions Commands
% ----------------------------------------------------------------------
\makeatletter
\ifdefined\theauthor
	\def\@author{\theauthor}
\else
	\def\author#1{\gdef\@author{#1}}
	\def\theauthor{\@author}
\fi
\def\titlefrench#1{\gdef\@titlefrench{#1}}
\def\thetitlefrench{\@titlefrench}
\def\titleenglish#1{\gdef\@titleenglish{#1}}
\def\thetitleenglish{\@titleenglish}
\def\keywordsfrench#1{\gdef\@keywordsfrench{#1}}
\def\thekeywordsfrench{\@keywordsfrench}
\def\keywordsenglish#1{\gdef\@keywordsenglish{#1}}
\def\thekeywordsenglish{\@keywordsenglish}
\def\advisor#1{\gdef\@advisor{#1}}
\def\theadvisor{\@advisor}
\def\degreeyear#1{\gdef\@degreeyear{#1}}
\def\thedegreeyear{\@degreeyear}
\def\degreemonth#1{\gdef\@degreemonth{#1}}
\def\thedegreemonth{\@degreemonth}
\def\degreeday#1{\gdef\@degreeday{#1}}
\def\thedegreeday{\@degreeday}
\makeatother


%
% Styling Commands
% ----------------------------------------------------------------------
%

\definecolor{chaptergrey}{rgb}{0.1000, 0.1600, 0.0060}%
\definecolor{hdrrule}{rgb}{0.4000, 0.4000, 0.4000}%

\fancypagestyle{empty}{
	\fancyhf{}
	\cfoot{}%\thepage}
	\renewcommand{\headrulewidth}{0pt}% header horizontal line
}

\fancypagestyle{emptynonum}{
	\fancyhf{}
	\cfoot{}
	\renewcommand{\headrulewidth}{0pt}% header horizontal line
}

\makeatletter
\@ifclasswith{book}{openany}{%
	\renewcommand*{\cleardoublepage}{\clearpage}%
}{%
	\renewcommand*{\cleardoublepage}{
		\clearpage
		\if@twoside
			\ifodd
				\c@page
			\else
				\hbox{}%
				\thispagestyle{empty}%
				
				\newpage%
				\if@twocolumn\hbox{}
					\newpage
				\fi
			\fi
		\fi
	}%
}
\makeatother

% page header
% ----------------------------------------------------------------------
\fancyhf{}% clear header
%\fancyhead[RO,LE]{\thepage}
%\fancyhead[LO]{\normalfont\nouppercase{\rightmark}}
%\fancyhead[RE]{\normalfont\nouppercase{\leftmark}}

\newcounter{realpagenum}
\fancyhead[RO,RE]{%
\setcounter{realpagenum}{\value{page}}%
\addtocounter{realpagenum}{\value{totalblankpages}}%
\ifodd\value{realpagenum}{\thepage}\else{\normalfont\nouppercase{\leftmark}}\fi%
}
\fancyhead[LO,LE]{%
\setcounter{realpagenum}{\value{page}}%
\addtocounter{realpagenum}{\value{totalblankpages}}%
\ifodd\value{realpagenum}{\normalfont\nouppercase{\rightmark}}\else{\thepage}\fi%
}

\pretocmd{\headrule}{\color{hdrrule}}{}{}
%\renewcommand{\headrulewidth}{0pt}% no horizontal line in headers
%\renewcommand{\footrulewidth}{0.4pt}% footer horizontal line
\pagestyle{fancy}
%
%\def\markchapter#1{\markboth{\MakeUppercase{#1}}{\MakeUppercase{#1}}}
\def\markchapter#1{\markboth{#1}{#1}}
\def\addchapter#1{\cleardoublepage\phantomsection{}\markchapter{#1}}% adds chapter but does not list it in toc
\def\addchaptertoc#1{\addchapter{#1}\addcontentsline{toc}{chapter}{#1}}% adds chapter and lists it in toc
\pretocmd{\tableofcontents}{\addchaptertoc{\contentsname}}{}{}
\newcommand{\listspart}{\addcontentsline{toc}{chapter}{Listes}}
\newif\ifListsStarted
\ListsStartedfalse
\newcommand{\listsname}{Listes}
\newcommand{\checkListsStarted}{\ifListsStarted\else\addcontentsline{toc}{chapter}{\listsname}\ListsStartedtrue\fi}
\pretocmd{\listoffigures}{\addchapter{\listfigurename}\checkListsStarted\addcontentsline{toc}{section}{\listfigurename}}{}{}
\pretocmd{\listoftables}{\addchapter{\listtablename}\checkListsStarted\addcontentsline{toc}{section}{\listtablename}}{}{}
\pretocmd{\printnomenclature}{\addchapter{\nomname}\checkListsStarted\addcontentsline{toc}{section}{\nomname}}{}{}
\newcommand\listacronymname{Liste des abréviations, sigles et acronymes}
\newcommand\listofacronyms{\addchapter{\listacronymname}\checkListsStarted\addcontentsline{toc}{section}{\listacronymname}\printglossary[type=\acronymtype,title=\listacronymname,style=mcoltree,nogroupskip]}
\pretocmd{\printindex}{\addchaptertoc{\indexname}}{}{}
\pretocmd{\bibliography}{\addchaptertoc{\bibname}}{}{}
\makeatletter
\newcounter{ChapterCounter}
\renewcommand{\theHsection}{\theChapterCounter.\the\value{section}}% used by hyperref, we modified this to make sure we provide unique string when using chapter*
\newif\ifChapterIsStarred
\renewcommand{\thesection}{\ifChapterIsStarred\else\the\value{chapter}.\fi\the\value{section}}% we modified this to hide chapter number when using chapter*
\def\startschapternotext#1{%
	\ChapterIsStarredtrue%
	\stepcounter{ChapterCounter}%
	\phantomsection\markchapter{#1}%
	\setcounter{section}{0}%
}
\pretocmd{\@schapter}{%
	\startschapternotext{#1}
}{}{}
\pretocmd{\@chapter}{%
	\ChapterIsStarredfalse%
	\stepcounter{ChapterCounter}%
}{}{}
\if@twoside
	% chapter title in even pages and section in odd pages
	\renewcommand{\chaptermark}[1]{\markboth{\chaptername\ \thechapter.\ #1}{\chaptername\ \thechapter.\ #1}}
	\renewcommand{\sectionmark}[1]{\markright{\thesection\ #1}}
\else
	% chapter title in both odd and even pages and no section
	\renewcommand{\chaptermark}[1]{\markboth{\chaptername\ \thechapter.\ #1}{\chaptername\ \thechapter.\ #1}}
	\renewcommand{\sectionmark}[1]{}
\fi
\makeatother

% front style
% ----------------------------------------------------------------------
\renewcommand{\frontmatter}{
	\pagenumbering{roman}
}

% main style
% ----------------------------------------------------------------------
\renewcommand{\mainmatter}{
	\cleardoublepage
	\setcounter{page}{1}
	\pagenumbering{arabic}
}

% back style
% ----------------------------------------------------------------------
\renewcommand{\backmatter}{
	\cleardoublepage
}

% etc
% ----------------------------------------------------------------------

\newcommand{\nullpart}{
	\bookmarksetup{startatroot}% Use bookmark package so that all following chapters appear in the root (no part)
}

\newenvironment{copyrightpage}{
	\clearpage{}% appears at the verso of the title page with year, author, the publisher, the ISSN number, etc. 
	\null\vspace*{\fill}
	\begin{center}
}{
	\end{center}
	\null
}

\newenvironment{dedications}{
	\cleardoublepage
	\null\vspace*{\stretch{1}}
	\begin{flushright}
	\large\itshape\myfancyfont
}{
	\end{flushright}
	\vspace*{\stretch{2}}\null
}

\newenvironment{abstract}{
	\par\noindent\textbf{\small\abstractname}%
	\par%
}{
}

\def\acknowledgmentsname{Remerciements}
\newenvironment{acknowledgments}{
	\chapter*{\acknowledgmentsname}
	\addcontentsline{toc}{chapter}{\acknowledgmentsname}
	\setstretch{1.2}
	\large\itshape
}{
}

\def\prefacename{Préface}
\newenvironment{preface}{
	\chapter*{\prefacename}
	\addcontentsline{toc}{chapter}{\prefacename}
}{
}

\def\listofpublicationsname{Nos publications}
\newenvironment{listofpublications}{
	\chapter*{\listofpublicationsname}
	\addcontentsline{toc}{chapter}{\listofpublicationsname}
}{
}

\newcommand{\inenglish}[1]{\textenglish{\textit{#1}}}% foreign langage for french should be in italic

\makeatletter
\def\setfigpath#1{\gdef\@figpath{#1}}
\def\figpath{\@figpath}
\makeatother

\captionsetup{%
	figurename=Figure,
	tablename=Tableau
}
%\AtBeginDocument{%
%\renewcommand\figurename{\textsc{Figure}}
%\renewcommand\tablename{\textsc{Tableau}}
%}

% Algorithms
% ----------------------------------------------------------------------
% numbering
\makeatletter 
\renewcommand\thealgorithm{\thechapter.\arabic{algorithm}} 
\@addtoreset{algorithm}{chapter} 
\makeatother

% locale
\makeatletter
\newcommand{\newalgname}[1]{%
  \renewcommand{\ALG@name}{#1}%
}
\newalgname{Algorithme}
\renewcommand{\listalgorithmname}{Liste des \ALG@name s}
\makeatother

\renewcommand{\algorithmicrequire}{\textbf{Entrée:}}
\renewcommand{\algorithmicensure}{\textbf{Sortie:}}
\renewcommand{\algorithmicend}{\textbf{fin}}
\renewcommand{\algorithmicif}{\textbf{si}}
\renewcommand{\algorithmicthen}{\textbf{alors}}
\renewcommand{\algorithmicelse}{\textbf{sinon}}
\renewcommand{\algorithmicfor}{\textbf{pour}}
\renewcommand{\algorithmicforall}{\textbf{pour chaque}}
\renewcommand{\algorithmicdo}{\textbf{faire}}
\renewcommand{\algorithmicwhile}{\textbf{tant que}}
\renewcommand{\algorithmicrepeat}{\textbf{répéter}}
\renewcommand{\algorithmicuntil}{\textbf{jusqu'à}}
\renewcommand{\algorithmicreturn}{\textbf{retourner}}
\newcommand{\algorithmicelsif}{\algorithmicelse\ \algorithmicif}
\newcommand{\algorithmicendif}{\algorithmicend\ \algorithmicif}
\newcommand{\algorithmicendfor}{\algorithmicend\ \algorithmicfor}


% other styling
% ----------------------------------------------------------------------

\newcommand{\enquote}[1]{«\,#1\,»}

\newcommand{\manualpagebreak}{\pagebreak}

% Copyright
% ----------------------------------------------------------------------
%\begin{copyrightpage}
%	\copyright{} \thedegreeyear{} \theauthor{}
%
%	TOUS DROITS RÉSERVÉS
%\end{copyrightpage}