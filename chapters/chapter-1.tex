% Chapitre 2
%
\chapter{Le titre du chapitre 1}
\setfigpath{chapter-1}
%
\section{Introduction}
Lorem ipsum dolor sit amet, consectetur adipiscing elit. Phasellus suscipit nulla et mattis mattis. Curabitur vel tempor tellus. 
Fusce lacinia fringilla fringilla. Vestibulum in nulla at dolor malesuada viverra. Morbi fermentum justo neque, quis aliquam quam vehicula eu. 
Aenean eu pharetra risus. Cras id magna et mi malesuada porttitor.

Aliquam varius, est et porttitor ultricies, massa mi rutrum velit, eu mattis felis leo in lorem. 
Aenean feugiat commodo ex, vitae feugiat purus semper eget. Praesent at arcu viverra, suscipit metus quis, blandit eros. 
Pellentesque a lacus sodales, lacinia ante at, condimentum eros. Nunc eu lacus tellus. In scelerisque eros ex, quis suscipit augue pharetra porta. 
Nam aliquam congue metus nec bibendum. Quisque sed auctor nibh.

Praesent consectetur est ut ultricies tempus. Integer placerat luctus lacinia. 
Vestibulum ante ipsum primis in faucibus orci luctus et ultrices posuere cubilia curae; Maecenas sapien nibh, egestas at eros vel, suscipit facilisis enim. 
Ut mollis eleifend ligula, quis sagittis enim consequat nec. Phasellus non lectus a nulla gravida interdum sit amet non erat. 
Nulla id luctus leo, sed consectetur magna. Cras tincidunt, lacus vel varius mollis, elit turpis finibus sem, ac viverra dolor mi id leo. 
In tellus lorem, cursus in iaculis vel, tristique vitae lacus. Aenean at ornare felis. Phasellus dapibus sit amet justo eu elementum. 
Orci varius natoque penatibus et magnis dis parturient montes, nascetur ridiculus mus. Quisque iaculis mauris at lorem venenatis hendrerit. 
Vestibulum purus diam, sagittis sed mauris a, congue efficitur lacus. Nullam suscipit justo et arcu porta lobortis.

\section{La sûreté de fonctionnement}
La sûreté de fonctionnement a été définie par J-C.~Laprie dans~\cite{laprie1985dependable} comme étant
\enquote{la propriété qui permet aux utilisateurs d'un système de placer une confiance justifiée dans le service qu'il leur délivre}.
Dans une autre définition~\cite{villemeur1988surete}, A.~Villemeur définit la sûreté de fonctionnement
comme étant \enquote{l'aptitude d'une entité à satisfaire à une ou plusieurs fonctions requises dans des conditions données}.

Nous présentons dans ce qui suit ses attributs, ses entraves et ses moyens.

\subsection{Attributs}
Lorem ipsum dolor sit amet, consectetur adipiscing elit. Phasellus suscipit nulla et mattis mattis. Curabitur vel tempor tellus. 
Fusce lacinia fringilla fringilla. Vestibulum in nulla at dolor malesuada viverra. Morbi fermentum justo neque, quis aliquam quam vehicula eu. 
Aenean eu pharetra risus. Cras id magna et mi malesuada porttitor.

\subsubsection{La fiabilité}
La fiabilité, en anglais \inenglish{reliability}, est la probabilité qu'a un système à remplir sa mission
dans un intervalle de temps donné et dans certaines conditions spécifiées dans
son cahier des charges.

Pour quantifier la fiabilité d'un système, il est courant de se référer
au temps moyen avant la première panne, noté $\mathit{MTTF}$ (de l'anglais \inenglish{Mean Time To Failure}).
Dans ce cas, si $\mathit{Rel}(t)$ représente la fiabilité du système durant
la durée $t$, alors l'équation suivante définit le $\mathit{MTTF}$ du système~:
\begin{equation}
\mathit{MTTF} = \int_{0}^{\infty} \mathit{Rel}(t) \diff t
\end{equation}

Lorsqu'un système est composé de plusieurs composants dont le $\mathit{MTTF}$ est
connu, alors le $\mathit{MTTF}$ de ce système est défini par l'équation suivante~:
\begin{equation}
\mathit{MTTF}_\text{système} = \frac{\mathit{MTTF}_\text{composant}}{\text{nombre~de~composants}}
\end{equation}

L'équation précédente implique que plus le nombre de composants constituant un système est grand, moins ce système
est fiable.

\subsubsection{La maintenabilité}
La maintenabilité, anglicisme de \inenglish{maintainability}, définit la probabilité qu'un
système en panne à un instant donné soit réparé après une certaine durée.

Cette aptitude est quantifiable en se référant au temps moyen avant la réparation,
noté $\mathit{MTTR}$ (de l'anglais \inenglish{Mean Time To Repair}).
Dans ce cas, si $\mathit{Mnt}(t)$ représente la probabilité qu'une réparation soit effectuée
durant la durée $t$, alors l'équation suivante définit le $\mathit{MTTR}$ du système~:
\begin{equation}
\mathit{MTTR} = \int_{0}^{\infty} (1 - \mathit{Mnt}(t)) \diff t
\end{equation}

Lorsque le système est réparable, il est courant d'utiliser le temps moyen
entre pannes, noté $\mathit{MTBF}$ (de l'anglais \inenglish{Mean Time Between Failures}),
au lieu du $\mathit{MTTF}$ pour quantifier la fiabilité du système.
Dans ce cas, le $\mathit{MTBF}$ mesure la durée moyenne entre deux
pannes (cf.\ \figurename~\ref{fig:tf_mttf}).

\begin{figure}
	\centering
	\includegraphics[scale=0.8]{\figpath/\figpref mttf.eps}
	\caption{\label{fig:tf_mttf}Séquence des événements de panne et de réparation dans un système réparable.}
\end{figure}

\subsection{Entraves}
Lorem ipsum dolor sit amet, consectetur adipiscing elit. Phasellus suscipit nulla et mattis mattis. Curabitur vel tempor tellus. 
Fusce lacinia fringilla fringilla. Vestibulum in nulla at dolor malesuada viverra. Morbi fermentum justo neque, quis aliquam quam vehicula eu. 
Aenean eu pharetra risus. Cras id magna et mi malesuada porttitor.

\subsection{Moyens}
Lorem ipsum dolor sit amet, consectetur adipiscing elit. Phasellus suscipit nulla et mattis mattis. Curabitur vel tempor tellus. 
Fusce lacinia fringilla fringilla. Vestibulum in nulla at dolor malesuada viverra. Morbi fermentum justo neque, quis aliquam quam vehicula eu. 
Aenean eu pharetra risus. Cras id magna et mi malesuada porttitor.

\section{Titre 2}
Lorem ipsum dolor sit amet, consectetur adipiscing elit. Phasellus suscipit nulla et mattis mattis. Curabitur vel tempor tellus. 
Fusce lacinia fringilla fringilla. Vestibulum in nulla at dolor malesuada viverra. Morbi fermentum justo neque, quis aliquam quam vehicula eu. 
Aenean eu pharetra risus. Cras id magna et mi malesuada porttitor.

\section{Conclusion}
Lorem ipsum dolor sit amet, consectetur adipiscing elit. Phasellus suscipit nulla et mattis mattis. Curabitur vel tempor tellus. 
Fusce lacinia fringilla fringilla. Vestibulum in nulla at dolor malesuada viverra. Morbi fermentum justo neque, quis aliquam quam vehicula eu. 
Aenean eu pharetra risus. Cras id magna et mi malesuada porttitor.

Aliquam varius, est et porttitor ultricies, massa mi rutrum velit, eu mattis felis leo in lorem. 
Aenean feugiat commodo ex, vitae feugiat purus semper eget. Praesent at arcu viverra, suscipit metus quis, blandit eros. 
Pellentesque a lacus sodales, lacinia ante at, condimentum eros. Nunc eu lacus tellus. In scelerisque eros ex, quis suscipit augue pharetra porta. 
Nam aliquam congue metus nec bibendum. Quisque sed auctor nibh.
